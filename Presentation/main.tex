\documentclass{beamer}
\usetheme{Boadilla}
\usepackage{xcolor, soul}
\sethlcolor{red}
\usepackage[absolute,overlay]{textpos}
\usepackage[ruled,vlined]{algorithm2e}
\usepackage{dsfont}
\usepackage{array}
\usepackage{tikz}
\usetikzlibrary{shapes, arrows.meta, positioning}
\usepackage[style=alphabetic]{biblatex}  
\addbibresource{lib.bib}
\usepackage{caption}
\usepackage{xcolor}
\usepackage{mathrsfs}



\author[Jonas Müller]{Jonas Carsten Reinhard Müller}


\setbeamertemplate{enumerate item}{\arabic{enumi}.}  % Use regular numbers and a period
%%%%%%%%%%%%%%%%%% Begin of the Presentation %%%%%%%%%%%%%%%%%

\begin{document}
\title[Wasserstein Metric]{The Metric side of Optimal Transport}
\institute{}
\date{}

\begin{frame}
    \titlepage
\end{frame}

%%%%%%%%%%%%%%%%%% Outline %%%%%%%%%%%%%%%%%%%%%

\section{A metric for probabilites?}
\subsection{The Wasserstein distance}
\subsection{What does this have to do with Optimal Transport?}
\subsection{It actually is a metric!}

\section{Lifting completeness from $X$ to $\mathscr{P}_2(X)$}

\section{Discussion}
\subsection{Outlook}


% table of contents 
\begin{frame}
    \frametitle{Outline}
    \tableofcontents
\end{frame}


%%%%%%%%%%%%%%%%%% Beginning of slides %%%%%%%%%%%%%%%%%


\begin{frame}{}
    start
\end{frame}

\begin{frame}{References}
    \printbibliography
\end{frame}

%%%%%%%%%%%%%%%%%% End of Presentation %%%%%%%%%%%%%%%%%


\end{document}